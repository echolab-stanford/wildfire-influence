\begin{table}[!htbp] \centering 
  \caption{\textbf{Counts of states in different total \pmt-trend classifications, under different sample restrictions and/or statistical specifications}. Main sample uses station-years that report at least 50 days in for 10 years, with the year break in 2016. Other samples are as listed, e.g. ``10yrs, 100obs" restricts to stations that report at least 100 days in each of at least 10 years. ``Drop" samples are those that drop individual years. ``No dup. break yr" is a sample that does not duplicate the break year. Specifications that identify a break year otherwise match the main specifcation, ``regional breaks" uses breaks specific to each region shown in Fig \ref{fig:breakpoints_regional}, and ``piecewise" forces segments on either side of the break year to intersect at the break year.} 
  \label{table:pmtrends} 
\footnotesize 
\begin{tabular}{@{\extracolsep{5pt}} cccccc} 
\\[-1.8ex]\hline 
\hline \\[-1.8ex] 
\textit{Specification} \\ 
                  \textit{or Sample} & reversal & stagnation & acceleration & non-sig. change & no sig. early decline \\ 
\hline \\[-1.8ex] 
Main & 17 & 25 & 0 & 4 & 2 \\ 
All obs & 14 & 27 & 0 & 5 & 2 \\ 
5yrs, 50obs & 16 & 26 & 0 & 4 & 2 \\ 
15yrs, 50obs & 16 & 27 & 0 & 1 & 4 \\ 
10yrs, 100obs & 17 & 25 & 0 & 2 & 4 \\ 
break in 2015 & 9 & 30 & 0 & 5 & 4 \\ 
break in 2017 & 11 & 27 & 0 & 5 & 5 \\ 
drop 2020 & 15 & 26 & 0 & 5 & 2 \\ 
drop 2021 & 10 & 28 & 1 & 7 & 2 \\ 
drop 2022 & 19 & 23 & 0 & 4 & 2 \\ 
no dup. break yr & 11 & 28 & 0 & 7 & 2 \\ 
piecewise & 10 & 30 & 1 & 4 & 3 \\ 
regional breaks & 14 & 28 & 0 & 5 & 1 \\ 
\hline \\[-1.8ex] 
\end{tabular} 
\end{table} 
